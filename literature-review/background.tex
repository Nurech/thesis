%! Author = partsjoo
%! Date = 06.04.2023

\newpage


\section{Background / State of the Art}

The rapid growth of technology, multimedia, and robotics has led to significant advancements in \ac{ict} infrastructure worldwide, prompting
the development of various educational programs. The evolution of technology has boosted the field of robotics, resulting in a wide array
of potential applications in \ac{he}. The use of robotics in education is increasing,
with \ac{tprs} being applied in \ac{hei}, and other diverse roles in the industry~\cite[]{telepresence_robots_in_classroom_2019,higher_edu_perception_on_tprs_2022}.

\ac{tprs} have great potential for pedagogic reasons within education at all levels, as they benefit \ac{he} personnel the replacement of
physical
presence and allow students with \ac{sen} have access to education they might miss otherwise due to their disabilities~\cite[546]{telepresence_robots_in_classroom_2019}.
\ac{tprs} have ability to create interaction between individuals which can be an opportunity for learning not only from a three-dimensional
inanimate
object but also through interaction with other people. This interaction enables \ac{tprs} to aid in improving social
skills in individuals with disabilities~\cite[541]{telepresence_robots_in_classroom_2019}.

Although the advantages of \ac{tprs} in education are numerous, this technology also creates new possible security risks that need to be
assessed. Interconnectivity with \ac{tprs} by the internet to the \ac{hei} means that the organization needs to be aware of possible
security risks~\cite[120]{robotics_cyber_security_2022}. Cybersecurity is crucial in \ac{hei} due to the vast amount of computing power and access to other resources
universities
have. These institutions hold large volumes of personal, financial, and intellectual data that can be attractive targets for cybercriminals.

It is inherently difficult to ensure security within robotics systems due to the complexity of robotic systems in general, which leads to
wide
attack surfaces
and a variety
of potential attack vectors~\cite[2]{robot_security_review_2022}. In addition, robotics manufactures often struggle to mitigate
vulnerabilities in reasonable time periods~\cite[12]{robot_security_review_2022}. The lack of investment in cybersecurity and the
immature state of the field in robotics cybersecurity contribute to the challenges in securing robotic systems~\cite[12]{
  robot_security_review_2022}. Most current robots are vulnerable, and defensive approaches
are
struggling
to keep up with the need for security~\cite[12]{robot_security_review_2022}. Therefore it is reasonable to assume that \ac{tprs} are also
vulnerable to similar security risks. Though there exists a veriety of risk assessment models, frameworks and methodolgies to assess
cybersecurity risks within robotics systems in general, the studies which focus on \ac{tprs} usage in \ac{hei} are limited and scattered.

Because of the lack of research on \ac{tprs} in \ac{hei} regarding cybersecurity risks, and the usage of \ac{tprs}
is increasing, this thesis aims to bridge the gap in the literature by providing a comprehensive review on the state of the art of
\ac{tprs} cybersecurity risks in \ac{hei} and offers possible mitigation strategies for identified cybersecurity risks.

\subsection{Related Work}

\subsection{Telepresence Robotics}

% Your content here

\subsection{Cyber Security Risks in Robotics}

% Your content here

\subsection{Existing Risk Assessment Models}

% Your content here
