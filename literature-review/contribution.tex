%! Author = partsjoo
%! Date = 05.04.2023

\newpage


\section{Contribution}

% Your content here

\subsection{Research method}

Primarily focus is on exploring and understanding the underlying perspectives, and experiences of the participants involved where \ac{tprs} have been deployed. Given
the nature of the study, the absence of empirical data, and the limited time for conducting case studies, a qualitative research approach
will help to gain deeper insights. Thus it is important to develop a review protocol which sets the framework for conducting a thorough and
unbiased review of the literature. It ensures that
systematic and rigorous approach is followed, which enhances the credibility and reliability of the review. Following review protocol helps
to minimize the risk of bias in the review process by establishing predefined criteria for study selection, quality assessment, and data
extraction.

% write enumerate with sections
\begin{enumerate}
  \item Establish the background and context of the study;
  \item Formulate clear and specific research questions;
  \item Define the search strategy, including search terms and resources to be used;
  \item Set the study selection criteria and procedures;
  \item Develop study quality assessment checklists and procedures;
  \item Design a data extraction strategy tailored to the research questions;
  \item Plan the synthesis of the extracted data, including descriptive and quantitative methods, as appropriate;
  \item Outline a dissemination strategy for the review findings;
  \item Set a project timetable to ensure timely completion of the review. \cite[4-5]{systematic_review_2004}
\end{enumerate}

\subsection{Search strategy}

Search strategy was developed to identify relevant literature for this review following the search strategy generation guidelines \cite[
  7-8]{systematic_review_2004}. Search strategy for the following thesis consists of 3 steps:

\begin{enumerate}
  \item Generation of keywords and search terms;
  \item The use of search filters;
  \item Selection of credible sources.
\end{enumerate}

\textbf{Keywords:} Initially, the search keywords were created by breaking down the research questions into their main concepts. A list of search terms and phrases related to each key concept was generated by applying term harvesting to each research question. Main terms were complemented by synonyms, alternative spellings, acronyms, and related terms to ensure a comprehensive search. To narrow down the search, the terms were combined using Boolean operators (where applicable). The result was a list of search terms and phrases used to query the databases.

\textbf{Filters:} The search filters were used to limit the search to the relevant studies. Most common filters used were: publication
year, language, and type of publication. The filters were applied to the search results to ensure that only relevant studies were
included in the review. Most important filter being the publication year. The search was limited to the last 10 years (2013-2023) to
ensure that only the most recent and relevant studies were included in the review. Primiary studies were limited to maximum age of 5
years and secondary studies to maximum age of 10 years.

\textbf{Sources:} To identify relevant research several databases were queried with the same search terms. Database selection was based
on the main category of hosted works (technology), the number of publications published and the age of the portal.
were:


\begin{table}[h]
  \centering
  \caption{Selected sources in order of relevance}
  \label{tab:sources}
  \begin{tabularx}{\textwidth}{|c|s|s|Y|c|}
    \hline
    {} & \heading{Publisher} & \heading{Metrics}   & \heading{Topics}                                                 & \heading{Foundation} \\
    \hline
    1  & SpringerLink        & 1,200 journals      & Computer science, engineering, environment                       & 1996                 \\
    \hline
    2  & IEEExplore          & 5,360,654 articles  & Computer science, electrical engineering and electronics  & 2000        \\
    \hline
    3  & Scopus              & 34,346 journals     & Life sciences, social sciences, physical sciences                & 2004                 \\
    \hline
    4  & ScienceDirect       & 15,000,000 articles & Physical Sciences and Engineering, Life Sciences  & 1997                 \\
    \hline
    5  & Web of Science      & 200 million records & Physical Sciences, Technology, Life Sciences \& Biomedicine  & 1998      \\
    \hline
    6  & LISTA               & 513 million records & Automation, Classification, Electronic resources and ERM systems  & 2005  \\
    \hline
    7  & Google Scholar      & 389 million records & Various topics                                                   & 2004                 \\
    \hline
  \end{tabularx}
\end{table}


After conducting preliminary searches using chosen search terms, and filters in the selected resources, search
queries were refined as needed to obtain required materials. Record the search strings used and the number of results obtained from each
resource was recorded for later use to check for updates on the subject.
