%! Author = partsjoo
%! Date = 06.04.2023

\newpage


\section{Contribution}\label{sec:contribution}

% Your content here

\subsection{Research method}\label{subsec:research-method}

Primarily focus is on exploring and understanding the cybersecurity risks, and the underlying perspectives of the participants
involved in \ac{hei} where \ac{tprs} have been deployed. Given
the nature of the study, the absence of known empirical data, and the limited time for conducting case studies, a qualitative research
approach
will help to gain deeper insights. Thus it is important to develop a review protocol which sets the framework for conducting a thorough and
unbiased review of the literature~\cite[8]{systematic_review_2004}. It ensures that
systematic and rigorous approach is followed, which enhances the credibility and reliability of the review. Following review protocol steps
helps
to minimize the risk of bias in the review process by establishing predefined criteria for study selection, quality assessment, and data
extraction:

% write enumerate with sections
\begin{enumerate}
  \item Establish the background and context of the study;
  \item Formulate clear and specific research questions;
  \item Define the search strategy, including search terms and resources to be used;
  \item Set the study selection criteria and procedures;
  \item Develop study quality assessment checklists and procedures;
  \item Design a data extraction strategy tailored to the research questions;
  \item Plan the synthesis of the extracted data, including descriptive and quantitative methods, as appropriate;
  \item Outline a dissemination strategy for the review findings;
  \item Set a project timetable to ensure timely completion of the review~\cite[4-5]{systematic_review_2004}.
\end{enumerate}

\subsection{Search strategy}\label{subsec:search-strategy}

Search strategy was developed to identify relevant literature for this review following the search strategy generation guidelines~\cite[7-8]{systematic_review_2004}. Search strategy for the following thesis consists of 3 steps:

\begin{enumerate}
  \item Generation of keywords and search terms;
  \item The use of search filters;
  \item Selection of credible sources.
\end{enumerate}

\textbf{Keywords:} Initially, the search keywords were created by breaking down the research questions into their main concepts. A list
of search terms and phrases related to each key concept was generated by applying term harvesting to each research question. Main keywords
were
complemented by secondary keywords (synonyms), alternative spellings, and related terms to ensure a comprehensive search. The
result was a list of
search terms and phrases used to query the databases. To narrow down the search, the terms were combined using Boolean operators (where
applicable).

\begin{table}[h]
  \centering
  \smaller
  \caption{Selected keywords and synonyms}
  \label{tab:keywords}
  \begin{tabular}{ll}
    \toprule
    \textbf{Primary keywords} & \textbf{Secondary keywords}  \\\midrule
    telepresence              & telerobotics, tele-education \\
    robotics                  & robot                        \\
    cybersecurity             & cyber, security, digital     \\
    risks                     & compromise, assessment       \\
    education                 & organization                 \\\bottomrule
  \end{tabular}
\end{table}

\textbf{Filters:} The search filters were used to limit the search to the relevant studies. Most common filters used were: publication
date, type of publication, discipline and language. The filters were applied to the search results to ensure that only relevant studies were
included in the review. Most important filter being the publication year. The search was limited to the last 10 years (2014-2023) to
ensure that only the most recent and relevant studies were included in the review. Primary studies were limited to maximum age of 5
years and secondary studies to maximum age of 10 years. Search filters of primary studies complemented the secondary seach filters.
Secondary studies were extended to other languages than English (incl. Estonian).

\begin{table}[h]
  \label{tab:filters}
  \smaller
  \caption{Used search filters}
  \centering
  \begin{tabularx}{400pt}{c>{\hsize=0.95\hsize}XXcXX}
    \toprule
    \multicolumn{3}{>{\hsize=3\hsize}{X}}{ \textbf{Primary studies}} & \multicolumn{3}{>{\hsize=3\hsize}{X}}{\textbf{Secondary studies$^{\ast}$ }} \\\midrule
    Date Range     & Type                                     & Discipline                    & Date Range     & Type                                         & Discipline                  \\\midrule
    2019\ldots2023 & Reports, Journals, Experiments, Datasets & Computer Science, Engineering & 2014\ldots2023 & Articles,Conference Proceedings, Whitepapers & Social Sciences, Psychology\\
    \bottomrule
  \end{tabularx}
  \begin{tabularx}{\textwidth}{@{}l}
    \footnotesize{$^{\ast}$ Includes primary search terms}\\
  \end{tabularx}
\end{table}

\textbf{Sources:}
After conducting preliminary searches using chosen search terms, and filters in the selected resources, search
queries were refined as needed to obtain required materials. Preliminary searches showed that using main keywords in serch strategy
produced large
number
of
results but highly relevant studies, thus the use of secondary keywords was not optimal in search for primary studies. Record the search
terms used and the number
of results
obtained from each
resource was recorded for later use to check for updates on the subject. Most filtering refinements were done in \ac{ui}. To identify relevant research several databases were queried with the same search terms. Database selection was based
on the main category of hosted works (technology), the number of publications published and the age of the portal.

\begin{table}[htb]
  \centering
  \smaller
  \caption{Selected sources in order of relevance}
  \label{tab:sources}
  \begin{tabularx}{\textwidth}{|c|c|c|c|Y|}
    \hline \textbf{} & \textbf{Publisher} & \textbf{Metrics}      & \textbf{Year} & \textbf{Topics}                                                  \\\hline
    1                & SpringerLink       & 1,200 journals        & 1996          & Computer science, Engineering, Environment                       \\\hline
    2                & IEEExplore         & 5,360,654 articles    & 2000          & Computer science, Electrical Engineering and Electronics         \\\hline
    3                & Scopus             & 34,346 journals       & 2004          & Life sciences, Social Sciences, Physical Sciences                \\\hline
    4                & ScienceDirect      & 15,000,000 articles   & 1997          & Physical Sciences and Engineering, Life Sciences                 \\\hline
    5                & Web of Science     & 200 million records   & 1998          & Physical Sciences, Technology, Life Sciences \&Biomedicine       \\\hline
    6                & LISTA              & 513 million records   & 2005          & Automation, Classification, Electronic resources and ERM systems \\\hline
    7                & Frontiers          & 185 academic journals & 2007          & Education, Computer Science, Robotics and AI                     \\\hline
    8                & Google Scholar     & 389 million records   & 2004          & Various topics                                                   \\\hline
  \end{tabularx}
\end{table}

Default language selection was English. Examples of search queries used with combination of search terms, filters on the selected
resources:
\begin{enumerate}
  \item SpringerLink -- \textit{"telepresence OR robots OR cybersecurity OR education OR risks"};
  \item IEEExplore -- \textit{"("All Metadata":telepresence) OR ("All Metadata":robots) OR ("All Metadata":cybersecurity) OR ("All Metadata"
  :education) OR ("All Metadata":risks)"};
  \item Scopus -- \textit{"TITLE-ABS-KEY ( telepresence OR robots OR cybersecurity OR education OR risks )  AND  ( LIMIT-TO ( PUBYEAR ,
    2023 )  OR LIMIT-TO ( PUBYEAR ,  2022 )  OR LIMIT-TO ( PUBYEAR ,  2021 )  OR LIMIT-TO ( PUBYEAR ,  2020 )  OR LIMIT-TO ( PUBYEAR
    ,  2019 ) )  AND  ( LIMIT-TO ( DOCTYPE ,  "ar" ) )  AND  ( LIMIT-TO ( SUBJAREA ,  "COMP" ) )"}.
\end{enumerate}

\begin{table}[htb]
  \centering
  \smaller
  \caption{Number of selected for review using primary search stategy}
  \label{tab:query-results}
  \begin{tabularx}{\textwidth}{|c|c|c|Y|Y|Y|Y|Y|}
    \hline
    \textbf{Publisher} & \textbf{Date} & \textbf{Range} & \textbf{Type} & \textbf{Discipline} & \textbf{Number of studies} \\\hline
    SpringerLink       & 03.2023       & 2019-2023      & Article       & Engineering         & 57349                      \\\hline
    IEEExplore         & 04.2023       & 2019-2023      & Journals      & Computer Science    & 61010                      \\\hline
    Scopus             & 04.2023       & 2019-2023      & Article       & Computer Science    & 101267                     \\\hline
    IEEExplore         & 04.2023       & 2019-2023      & All           & X                   & 221674                     \\\hline
    SpringerLink       & 03.2023       & 2019-2023      & All           & X                   & 1649397                    \\\hline
    Scopus             & 04.2023       & 2019-2023      & All           & X                   & 7865145                    \\\hline
  \end{tabularx}
\end{table}

\ldots

\subsection{Selection of primary and secondary studies}\label{subsec:selection-of-studies}

To identify primary studies that provide direct evidence about the research question, specific selection criteria was defined during the
protocol definition stage. Although criteria was refined during the search process, it served as a foundation for identifying
relevant studies. The selection criteria included factors such as study design, levels of evidence, and outcome measures, ensuring that the
chosen studies directly addressed our research question ~\cite[10-16]{systematic_review_2004}. \ac{tprs} in the context of cybersecurity
and education is a relatively new field of robotics with most research starting from 2̃015. Therefore, the selection of primary
studies was
extended to include secondary studies that provide indirect evidence about the research question.

In the context of assessing cybersecurity risks in telepresence robotics for higher education systems, the literature analysis process
identified \# key findings that directly influence the assessment of cybersecurity risks in \ac{hei} or in other ways support the
answering of research questions. Among
these studies, it was taken into account that secondary studies within the timerange of 2014\ldots2023 may have outdated information and
as such their findings were noted and used as supportive information. Throughout the discussion of each study, the aspects related to
cyber security risks in \ac{tprs} are highlighted. Table \# lists findings and the corresponding references:

\begin{enumerate}
  \item Identification of security risks associated with \ac{tprs}.
  \item Analysis of existing risk assessment models and their limitations in addressing \ac{tprs}-specific security challenges.
  \item Potential security vulnerabilities in \ac{tprs}, including hardware, software, and network-related issues.
  \item Importance of \ac{tprs} in \ac{hei} and their.
  \item Examination of studies that highlight \ac{tprs} security incidents and the effectiveness of implemented countermeasures.
  \item Development of guidelines and recommendations for organizations to better understand and manage \ac{tprs} security risks.
  \item Exploration of future research directions to enhance the security of \ac{tprs} in higher education and other settings.
  \item Investigation of mitigation strategies and best practices to address identified weaknesses in \ac{tprs} security.
\end{enumerate}

\begin{table}[h!]
  \centering
  \smaller
  \caption{Selected primary and secondary studies}
  \label{tab:studies}
  \begin{tabularx}{\columnwidth}{lcll}
    \toprule
    \textbf{Outcome measures}                                                                       & \textbf{Year} & \textbf{Study design} & \textbf{Evidence Level} \\
    \midrule
    Perceived usefulness of \ac{tprs}~\cite[]{acceptance-telepresence-robots-2022}                  & 2022          & Quasi-random          & 4-3                     \\
    Robotics cybersecurity and countermeasures~\cite[]{cyber_security_issues_in_robotics_2021}                                     & 2021          & Expert opinion        & 5                       \\
    Vulnerabilities and security solutions in robotics domain~\cite[]{robotics_cyber_security_2022}                                  & 2022& Expert opinion& 5                       \\
    Smart design engineering~\cite[]{smart_design_engineering_2020}                                 & 2020          & Expert opinion& 5                       \\
    Methodology to protect robots~\cite[]{robot_security_review_2022}                               & 2022          & Expert opinion& 5                       \\
    Perspectives of implementing \ac{tprs}~\cite[]{higher_edu_perception_on_tprs_2022}                                                         & 2022          & Quasi-random          & 4-3                     \\
    Methodology to perform security assessments in robotics~\cite[]{robot_security_framework_2018}                                         & 2018          & Expert opinion& 5                       \\
    Security measures in \ac{he}~\cite[]{role_of_cyber_security_in_higher_edu_2020}                 & 2020          & Expert opinion& 5                       \\
    Telepresence services in the \ac{hei}~\cite[]{telepresence_robots_in_classroom_2019}                                                           & 2019          & Expert opinion& 5                       \\
    \bottomrule
  \end{tabularx}
\end{table}

\subsection{Extracted data}\label{subsec:extracted-data}

todo
