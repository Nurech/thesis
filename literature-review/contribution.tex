%! Author = partsjoo
%! Date = 06.04.2023

\newpage


\section{Contribution}\label{sec:contribution}

% Your content here

\subsection{Research method}\label{subsec:research-method}

Primarily focus is on exploring and understanding the cybersecurity risks, and the underlying perspectives of the participants
involved in \ac{hei} where \ac{TPR} have been deployed. Given
the nature of the study, the absence of known empirical data, and the limited time for conducting case studies, a qualitative research
approach
will help to gain deeper insights. Thus it is important to develop a review protocol which sets the framework for conducting a thorough and
unbiased review of the literature~\cite[8]{systematic_review_2004}. It ensures that
systematic and rigorous approach is followed, which enhances the credibility and reliability of the review. Following review protocol steps
helps
to minimize the risk of bias in the review process by establishing predefined criteria for study selection, quality assessment, and data
extraction:

% write enumerate with sections
\begin{enumerate}
  \item Establish the background and context of the study;
  \item Formulate clear and specific research questions;
  \item Define the search strategy, including search terms and resources to be used;
  \item Set the study selection criteria and procedures;
  \item Develop study quality assessment checklists and procedures;
  \item Design a data extraction strategy tailored to the research questions;
  \item Plan the synthesis of the extracted data, including descriptive and quantitative methods, as appropriate;
  \item Outline a dissemination strategy for the review findings;
  \item Set a project timetable to ensure timely completion of the review~\cite[4-5]{systematic_review_2004}.
\end{enumerate}

\subsection{Search strategy}\label{subsec:search-strategy}

Search strategy was developed to identify relevant literature for this review following the search strategy generation guidelines~\cite[7-8]{systematic_review_2004}. Search strategy for the following thesis consists of 3 steps:

\begin{enumerate}
  \item Generation of keywords and search terms;
  \item The use of search filters;
  \item Selection of credible sources.
\end{enumerate}

\textbf{Keywords:} Initially, the search keywords were created by breaking down the research questions into their main concepts. A list
of search terms and phrases related to each key concept was generated by applying term harvesting to each research question. Main keywords
were
complemented by secondary keywords (synonyms), alternative spellings, and related terms to ensure a comprehensive search. The
result was a list of
search terms and phrases used to query the databases. To narrow down the search, the terms were combined using Boolean operators (where
applicable).

\begin{table}[h]
  \centering
  \caption{Selected keywords and synonyms}
  \label{tab:keywords}
  \begin{tabular}{|l|l|}
    \hline
    \textbf{Primary keywords} & \textbf{Secondary keywords}     \\\hline
    telepresence              & telerobotics, tele-education    \\\hline
    robotics                  & robot                           \\\hline
    cybersecurity             & cyber, security, digital        \\\hline
    risks                     & compromise, assessment, threats \\\hline
    education                 & organization                    \\\hline
  \end{tabular}
\end{table}

\textbf{Filters:} The search filters were used to limit the search to the relevant studies. Most common filters used were: publication
date, type of publication, discipline and language. The filters were applied to the search results to ensure that only relevant studies were
included in the review. Most important filter being the publication year. The search was limited to the last 10 years (2014-2023) to
ensure that only the most recent and relevant studies were included in the review. Primary studies were limited to maximum age of 5
years and secondary studies to maximum age of 10 years. Search filters of primary studies complemented the secondary seach filters.
Secondary studies were extended to other languages than English (incl. Estonian).

\begin{table}[h!]
  \label{tab:filters}
  \caption{Used search filters}
  \centering
  \begin{tabularx}{\textwidth}{|l|X|X|l|X|X|}
    \hline
    \colThree{Primary studies} & \colThreeEnd{Secondary studies$^{\ast}$ } \\\hline
    Date Range & Type & Discipline & Date Range & Type & Discipline \\\hline
    2019\ldots2023 & Reports, Journals, Experiments, Datasets & Computer Science, Engineering & 2014\ldots2023 & Articles, Conference
    Proceedings, Whitepapers & Social Sciences, Psychology \\\hline
  \end{tabularx}
  \begin{tabularx}{\textwidth}{X}
    \footnotesize{$^{\ast}$ Includes primary search terms}\\
  \end{tabularx}
\end{table}

\textbf{Sources:}
After conducting preliminary searches using chosen search terms, and filters in the selected resources, search
queries were refined as needed to obtain required materials. Preliminary searches showed that using main keywords in serch strategy
produced large
number
of
results but highly relevant studies, thus the use of secondary keywords was not optimal in search for primary studies. Record the search
terms used and the number
of results
obtained from each
resource was recorded for later use to check for updates on the subject. Most filtering refinements were done in \ac{ui}. To identify relevant research several databases were queried with the same search terms. Database selection was based
on the main category of hosted works (technology) as indicators show that most \ac{TPR} related materials originate from \ac{CS} field~\cite[62]{telepresence_perspective_psychology_educational_2022},
the number
of publications published
and
the
age
of
the portal. In the
execution phase
main databases were (SpringerLink, IEEExplore, Scopus, ScienceDirect, Web of Science) due to their large number of publications and
relvance on the topic. Initial queries yeilded small number of results, thus the search was extended to secondary sources (LISTA, Frontiers, Google Scholar).

\begin{table}[h!]
  \centering
  \caption{Selected sources in order of relevance}
  \label{tab:sources}
  \begin{tabularx}{\textwidth}{|c|c|c|Y|}
    \hline \textbf{} & \textbf{Publisher} & \textbf{Year} & \textbf{Topics}                                            \\\hline
    1                & SpringerLink       & 1996          & Computer Sciences, Engineering, Environment                \\\hline
    2                & IEEExplore         & 2000          & Computer Sciences, Electrical Engineering and Electronics  \\\hline
    3                & Scopus             & 2004          & Physical Sciences, Life sciences, Social Sciences          \\\hline
    4                & ScienceDirect      & 1997          & Physical Sciences and Engineering, Life Sciences           \\\hline
    5                & Web of Science     & 1998          & Physical Sciences, Technology, Life Sciences \&Biomedicine \\\hline
    6 $^{\ast}$      & LISTA              & 2005          & Classification, Electronic resources and ERM systems       \\\hline
    7 $^{\ast}$      & Frontiers          & 2007          & Education, Computer Science, Robotics and AI               \\\hline
    8 $^{\ast}$      & Google Scholar     & 2004          & Various topics                                             \\\hline
    9 $^{\ast}$      & Research Gate      & 2008          & Computer Sciences, Engineering, Social Sciences            \\\hline
  \end{tabularx}
  \begin{tabularx}{\textwidth}{X}
    \footnotesize{$^{\ast}$ Secondary sources}\\
  \end{tabularx}
\end{table}

Search scope default language selection was English. Examples of search search string composition used with combination of search terms,
filters on
the
selected resources can be seen in Table~\ref{tab:search_string}. The search string used all possible combinations. Preliminary
searches showed
that using main
keywords yielded best results.

\begin{table}[h!]
  \centering
  \caption{Composition of search strings}
  \label{tab:search_string}
  \begin{tabularx}{300pt}{|c|X|}
    \hline \textbf{} & \colOne{Search string} \\\hline
    1 & telepresence AND cybersecurity OR risks AND assessment AND robots AND
    education \\\hline
    2 & tele-robotics AND cybersecurity AND risks AND robots OR education AND
    assessment AND compromise OR mitigation \\\hline
    3 & telepresence AND cybersecurity AND risks AND assessment AND robots AND
    education AND management AND mitigation AND security AND compromise OR threats \\\hline
  \end{tabularx}
\end{table}

The databases used in this study contained scientific journals, conference proceedings, books, and trade journal articles. In April 2023, the database search was conducted, yielding a total
of 927 publications. Some references were found in multiple databases. To eliminate duplicates, a unified list featuring the titles of
the chosen publications was created. The abstracts of these publications were reviewed to confirm their relevance to \ac{TPR} and
cybersecurity. After
implementing this exclusion criterion, 19 publications were chosen for more in-depth analysis as seen in
Table~\ref{tab:query_results}.

\begin{table}[h!]
  \centering
  \caption{Number of selected studies found and selected for analysis}
  \label{tab:query_results}
  \begin{tabular}{|l|l|l|l|l|l|l|l|l|l|}
    \hline
    \colTwo{} & \colTwoEnd{Journal} & \colTwoEnd{Conference Papers} & \colTwoEnd{Book Chapter} & \colTwoEnd{Whitepaper} \\\hline
    \textbf{Year}   & \textbf{Total} & \centre{T} & \centre{S} & \centre{T} & \centre{S} & \centre{T}          & \centre{S}            & \centre{T}            & \centre{S}            \\\hline
    2014            & 14             & 6          &            & 6          &            & 2          &            &            &            \\\hline
    2015            & 10             & 5          &            & 3          &            & 2          &            &            &            \\\hline
    2016            & 7              & 6          & 1          &            &            & 1          &            &            &            \\\hline
    2017            & 18             & 12         & 2          & 5          &            & 1          &            &            &            \\\hline
    2018            & 32             & 22         & 1          & 3          &            & 6          &            & 1          &            \\\hline
    2019            & 63             & 39         & 1          & 15         &            & 9          &            &            &            \\\hline
    2020            & 104            & 68         & 2          & 16         &            & 20         &            &            &            \\\hline
    2021            & 250            & 122        & 5          & 40         & 1          & 88         &            &            &            \\\hline
    2022            & 306            & 116        & 5          & 59         &            & 131        &            &            & 1          \\\hline
    2023            & 123            & 54         &            & 16         &            & 52         &            & 1          &            \\\hline
    \textbf{Total}: & 927            & 450        & 17         & 163        & 1          & 312        &            & 2          & 1          \\\hline
  \end{tabular}
  \begin{tabularx}{\textwidth}{X}
    \footnotesize{$^{\ast}$ T - total, S - selected}\\
  \end{tabularx}
\end{table}


\subsection{Selection of primary and secondary studies}\label{subsec:selection-of-studies}

To identify primary studies that provide direct evidence about the research question, specific selection criteria was defined during the
protocol definition stage. Although criteria was refined during the search process, it served as a foundation for identifying
relevant studies. The selection criteria included factors such as study design, levels of evidence, and outcome measures, ensuring that the
chosen studies directly addressed our research question ~\cite[10-16]{systematic_review_2004}. \ac{TPR} in the context of cybersecurity
and education is a relatively new field of robotics with most research starting from 2̃015. Therefore, the selection of primary
studies was
extended to include secondary studies that provide indirect evidence about the research question.

In the context of assessing cybersecurity risks in telepresence robotics for higher education systems, the literature analysis process
identified 7 key findings that directly influence the assessment of cybersecurity risks in \ac{hei} or in other ways support the
answering of research questions. Among
these studies, it was taken into account that secondary studies within the timerange of 2014\ldots2023 may have outdated information and
as such their findings were noted and used as supportive information. Throughout the discussion of each study, the aspects related to
cyber security risks in \ac{TPR} are highlighted. Table~\ref{table:selected_sources} lists findings and the corresponding references:

\begin{enumerate}
  \item Identification of security risks associated with \ac{TPR}. %1 security risks
  \item Analysis of existing risk assessment models and their limitations in addressing \ac{TPR}-specific security challenges. %2 risk models
  \item Potential security vulnerabilities in robotics, including hardware, software, and network-related issues. %3 TRPs vulnerabilities
  \item Usage of \ac{TPR} in \ac{hei}. %4 TRPS and EDU
  \item Examination of studies that highlight \ac{TPR} security incidents and the effectiveness of implemented countermeasures. %5 countermeasures
  \item Exploration of future research directions to enhance the security of \ac{TPR} in higher education and other settings. %6 future research
  \item Investigation of mitigation strategies and best practices to address identified weaknesses in \ac{TPR} security. %7 mitigation strategies
\end{enumerate}

\begin{table}[h!]
  \centering
  \caption{Selected sources}
  \label{table:selected_sources}
  \begin{tabularx}{\textwidth}{|l|l|X|l|l|l|l|l|l|l|l|l|}
    \hline
    \textbf{\#}  & \textbf{Year} & \textbf{References}                                                         & \textbf{1} & \textbf{2} & \textbf{3} & \textbf{4} & \textbf{5}& \textbf{6}& \textbf{7}  \\\hline
    1            & 2021          & Zhu et al.~\cite[]{introduction_to_robot_system_security_2021}              & X          & X          &            &            & X          &            & X          \\\hline
    2            & 2021          & Fosch-Villaronga et al.~\cite[]{cyber_sec_safet_robots_legal_2021}          & X          & X          &            &            &  &  & X \\\hline
    3            & 2022          & Verizon~\cite[]{dbir_2022}                                                  &            &            &            &            &            & X          & X          \\\hline
    4            & 2021          & Lacava et al.~\cite[]{cyber_security_issues_in_robotics_2021}               & X          &            &            &            &            & X          &            \\\hline
    5            & 2022          & Yaacoub et al.~\cite[]{robotics_cyber_security_2022}                        & X          & X          & X          &            & X          & X          & X          \\\hline
    6            & 2020          & Pessoa et al.~\cite[]{smart_design_engineering_2020}                        &            &            &            &            & X          & X          &            \\\hline
    7            & 2021          & Villaronga et al.~\cite[]{cyber_sec_safety_robots_legal_2021}               & X          & X          &            &            &            &            & X          \\\hline
    8 $^{\ast}$  & 2017          & Lera et al.~\cite[]{cyber_sec_robotics_privacy_safety_2017}                 & X          & X & X &  &  & X &  \\\hline
    9 $^{\ast}$  & 2016          & Sabine et al.~\cite[]{if_robots_cause_harm_2016}                            &            &            & X          &            &            &            &            \\\hline
    10 $^{\ast}$ & 2018          & Ahmad et al.~\cite[]{analyzing_cyber_physical_threats_2018}                 & X          & X & X & X &  &  & X \\\hline
    11 $^{\ast}$ & 2017          & Portugal et al.~\cite[]{role_of_security_in_human_robot_2017}               & X &  & X &  &  &  &  \\\hline
    12 $^{\ast}$ & 2022          & Lei et al.~\cite[]{acceptance_telepresence_robots_2022}                     & X          &            &            & X          &  &  &  \\\hline
    13 $^{\ast}$ & 2022          & Leoste et al.~\cite[]{higher_edu_perception_on_tprs_2022}                   &            &            &            & X &  &  &  \\\hline
    14 $^{\ast}$ & 2018          & Vilches et al.~\cite[]{robot_security_framework_2018}                       & X          & X          &            &            &            &            &            \\\hline
    15 $^{\ast}$ & 2022          & Mayoral-Vilches~\cite[]{robot_security_review_2022}                         &            &            & X          &            & X          &            &            \\\hline
    16 $^{\ast}$ & 2020          & Singar et al.~\cite[]{role_of_cyber_security_in_higher_edu_2020} &  &  &  & X &  &  & X \\\hline
    17 $^{\ast}$ & 2019          & Reis et al.~\cite[]{telepresence_robots_in_classroom_2019}                  &            &            &            & X &  &  &  \\\hline
    18 $^{\ast}$ & 2022          & Virkus et al.~\cite[]{telepresence_perspective_psychology_educational_2022} &  &  &  & X &  &  &  \\\hline
    19 $^{\ast}$ & 2021          & Gupta et al.~\cite[]{humans_and_robots_relation_2021}                       &            &            &            & X          &            &            &            \\\hline

  \end{tabularx}
  \begin{tabularx}{\textwidth}{X}
    \footnotesize{$^{\ast}$ Secondary studies}\\
  \end{tabularx}
\end{table}

\newpage
\subsection{Extracted data}\label{subsec:extracted-data}

% content here
