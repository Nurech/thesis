\section{Propose the privacy-by-design recommendations what should be done to
increase compliance and decrease data leakage}

To increase compliance and decrease data leakage, it is recommended to implement
a privacy-by-design approach that includes implementing technical and
organisational measures, regularly assessing data protection risks, and
verifying legal grounds for data
processing~\cite[110-111]{10.1007/978-3-030-58135-0_9}. In this case, we are
using the DPO tool which can offer guidance to achieve GDPR
compliance~\cite{dpotool}.

To achieve GDPR compliance, we annotated and made flow additions to the initial
model with the following ideas in mind:
\begin{enumerate}
  \item Prompt to introduce the \textbf{Privacy Policy} to the user (storage
  period, right to access, legal basis, \dots);
  \item Treat PSP as the \textbf{controller} and assign resulting obligations
  for data (confidentiality, integrity, availability, \dots);
  \item \textbf{Consent} needs to be taken from the user (clear purpose,
  unambiguous, affirmative action, \dots);
  \item Store a \textbf{record of document processing} with (purpose, contact
  details, data storage period, \dots).
\end{enumerate}

Figure~\ref{fig:improved-model} shows our improved and annotated model.

\begin{landscape}

\begin{figure}[ht]
\begin{center}
  \includegraphics[height=\textwidth - 137pt]{improved.pdf}
  \caption{Improved BPMN model (with GDPR annotations)}
  \label{fig:improved-model}
\end{center}
\end{figure}

\end{landscape}

After performing a GDPR compliance analysis with the DPO tool on the revised
model, we obtained satisfactory results---all initial shortcomings were
addressed. Figure~\ref{fig:improved-uml} displays the result of the updated
evaluation.

\begin{figure}[!hb]
\begin{center}
  \includegraphics[width=\textwidth - 32pt]{improved-uml.pdf}
  \caption{DPO tool evaluation of the corrected model}
  \label{fig:improved-uml}
\end{center}
\end{figure}
