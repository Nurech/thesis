%! Author = partsjoo
%! Date = 16.04.2023

\newpage
\section{Introduction}

% Your content here

\subsection{Problem Statement}

With the increasing popularity of \ac{TPR} in higher education systems, enabling users to remotely partake in events, new security risks
which could be characterized specific to \ac{TPR} have emerged~\cite{acceptance_telepresence_robots_2022,
  cyber_security_issues_in_robotics_2021,robotics_cyber_security_2022,robot_security_review_2022}.
These risks include abuse of privilege, unauthorized access, cyber-physical risks, and exposure of sensitive data among other risks~\cite[
  120]{robotics_cyber_security_2022}.
However, current risk assessment models do not adequately address these unique concerns, resulting in a knowledge gap in understanding 
and mitigating TPR-related risks~\cite[]{robotics_cyber_security_2022}.
This master's thesis aims to explore potential security issues associated with \ac{TPR} and propose methods to mitigate them by
reviewing the state of art, conducting case studies and interviewing experts.

The research will involve identifying and validating potential risks through case studies and expert interviews, with a focus on identifying
cybersecurity risks \ac{TPR} may pose to higher education systems.
By proposing mitigation strategies and emphasizing cybersecurity, this study seeks to provide organizations utilizing \ac{TPR} with a
better
understanding of security risks and effective solutions to protect their systems and users.
Ultimately, this research will contribute to bridging the gap between existing knowledge and the unique security concerns presented by the
growing use of \ac{TPR} in \ac{HEI}.

\subsection{Objectives and Roadmap}\label{subsec:objectives-and-roadmap}

% Your content here

\subsubsection{Research Objective}

The primary objective of this thesis is to identify cybersecurity risks related to organizations using \ac{TPR} and propose mitigation strategies to reduce the identified risks.

\subsubsection{Research Questions}

\textbf{RQ: To what kind of security risks are organisations using \ac{TPR} exposed to, and how to mitigate the risks?}


The main research question is divided into three research subquestions, the potential security risks posed by \ac{TPR}, how organisations
have assessed the potential risks and the solutions that can be provided to reduce the identified risks. The following subquestions are formulated
in sequential order according to their importance:

\begin{enumerate}
  \item\textbf{RQ1}: What are the potential security risks posed by \ac{TPR}, and how do these risks uniquely impact organizations utilizing these systems?
  \item\textbf{RQ2}: How have organizations implemented assessment and management strategies to address cybersecurity risks associated with telepresence robotics?
  \item\textbf{RQ3}: What potential solutions can be provided to reduce identified security risks?
\end{enumerate}

\textbf{RQ1}: Identification of potential security risks posed by \ac{TPR} is the first step. In this step the possible security risks will be identitified by analyzing existing frameworks, migigation strategies and previous works in the field. This sub-research question focuses on uncovering the distinct security risks associated with telepresence robotics and examines their implications for organizations that deploy \ac{TPR}.
  By identifying these risks, the research will contribute to a comprehensive understanding of the challenges and vulnerabilities that need to be addressed in order ensure secure operation of \ac{TPR} systems.
  This exploration will consider various aspects of \ac{TPR}, such as remote connectivity, cyber-physical presence, and live video and audio feeds, to highlight the unique security concerns that arise from their use.
  Additionally, the research will investigate how these risks may differ from those faced by organizations using other types of robotics and what factors contribute to the increased vulnerability of \ac{TPR} systems.

\textbf{RQ2}: Once we have identified possible security risks the next step is to examine how have organizations implemented assessment
and management strategies to address cybersecurity risks associated with \ac{TPR}? This sub-research question focuses on understanding
the mechanisms and processes involved in managing \ac{TPR} systems. Finding the issues and gaps in current implementation is important to validate found security risks from teoretical material, but is also important before appropriate mitigation strategies can be considered. Addressing this question is essential for identifying potential vulnerabilities and areas where security improvements can be made.

\textbf{RQ3}: Following the identification of security risks associated with \ac{TPR}, this sub-research question concentrates on investigating and proposing potential mitigation strategies that effectively address the recognized risks. The study will explore a range of solutions, including technological advancements, policy implementation, and organizational practices, to provide a comprehensive understanding of how organizations can secure their \ac{TPR} systems. The proposed solutions should be practical, effective to the needs of organizations using \ac{TPR}.
  This will involve considering the unique security risks posed by \ac{TPR} and the distinct contexts in which they are deployed.
  The focus will be on the interaction between external users and \ac{TPR}, ensuring that the proposed mitigation strategies safeguard sensitive information and maintain the privacy and security of all parties involved.


By addressing these three sub-research questions, the thesis aims to provide a comprehensive understanding of the security risks faced by
organizations using \ac{TPR} and offer practical solutions for mitigating these risks, ultimately contributing to a more secure and
reliable \ac{TPR} environment.

\subsubsection{Roadmap and Structure}

To achieve the research objective, the following roadmap and structure will be followed as also seen on Figure~\ref{fig:flowdiagram}:

\begin{enumerate}
  \item Literature Review and analysis: A comprehensive review of existing research on \ac{TPR}, risk assessment models, and related
  frameworks will be conducted to identify potential issues within \ac{TPR} systems.
  \item Case Studies: Case studies will be conducted to validate existence of possible security risks by analyzing real-life scenarios
  involving \ac{TPR} usage in organizations;
  \item Expert Interviews: Interviews with technical staff who have experience in integrating \ac{TPR} into organizations will be
  conducted to confirm the identified risks and explore possible mitigation strategies proposed by the experts;
  \item Data Analysis and Proposed Mitigation Strategies: The findings from case studies and expert interviews will be analyzed to
  identify potential security concerns and risks posed by \ac{TPR}, as well as potential solutions to these risks;
  \item Conclusion: The thesis will conclude by summarizing the key findings, discussing the limitations of the research, and suggesting avenues for future research.
\end{enumerate}
Following this roadmap, the thesis will contribute to bridging the gap between existing cybersecurity knowledge regarding robotics in
higher education systems and the unique security concerns presented by the growing use of \ac{TPR}.

%! Author = partsjoo
%! Date = 16.04.2023

\begin{figure}
  \centering
  \begin{tikzpicture}[scale=0.8,
  box/.style={rectangle, draw, fill=gray!20, text width=2.3cm, minimum height=1.3cm, align=center},
  arrow/.style={->, thick, >=stealth},
  node distance=0.2cm and 0.3cm
  ]

  \node[box] (box1) {1. Build theoretical background. Identify potential risks.};
  \node[box, right=of box1] (box2) {2. Validate risks using case studies, gain deeper insight.};
  \node[box, right=of box2] (box3) {3. Conduct interviews. Experts confirm risks and provide mitigation strategies.};
  \node[box, right=of box3] (box4) {4. Risk and data analysis};
  \node[box, right=of box4] (box5) {5. Conclusion, key findings, future research};

  \draw[arrow] (box1) -- (box2);
  \draw[arrow] (box2) -- (box3);
  \draw[arrow] (box3) -- (box4);
  \draw[arrow] (box4) -- (box5);
  \draw[arrow] (box5) -- ++ (0,-2.7) -| (box1.south);

  \end{tikzpicture}
  \caption{The initial plan for Master Project.}
  \label{fig:flowdiagram}
\end{figure}


\subsection{Preliminaries}

% Your content here

