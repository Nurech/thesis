%! Author = partsjoo
%! Date = 05.04.2023

\newpage
\noindent\textbf{\large {\thesisTitle}}

\vspace*{3ex}

\noindent\textbf{Abstract:}
\vspace{-1ex}

\ac{tprs} have become increasingly popular, particularly
in higher education systems, as they enable users to remotely partake in
events. However, this increased usage also presents potential security risks
specific to \ac{tprs}, such as remote connections, cyber-physical presence, and
live video and audio feeds. Current risk assessment models do not adequately
address these unique concerns, leading to a gap in understanding and
mitigating \ac{tprs} related risks.
This thesis aims to map potential security issues, offer mitigation strategies for found weaknesses, and bridges the gap by
conducting case
studies and expert interviews.
This research will provide organizations utilizing \ac{tprs} with a
better understanding of security risks and effective solutions to protect
their systems and users.

\vspace*{1ex}

\noindent\textbf{Keywords:}
\vspace{-1ex}
{\thesisKeywordsEng}
\vspace*{1ex}

\noindent\textbf{CERCS:}
\vspace{-1ex}
{\thesisCERCSEng}
\vspace*{1ex}

\noindent\textbf{\large {\thesisTitleEst}}
\vspace*{1ex}

\noindent\textbf{Lühikokkuvõte:}
\vspace{-1ex}

Kaugosalus robotid on muutunud üha populaarsemaks, eriti kõrgemas haridussüsteemis, kuna need võimaldavad kasutajatel osaleda üritustel
kaugjuhtimise teel. Siiski kaasnevad selle suurenenud kasutamisega ka kaugosalus robotitele omased potentsiaalsed turvariskid, nagu
kaugühendus, küber-füüsiline kohalolek ning reaalajas video- ja heliside. Praegused riskihindamise mudelid ei käsitle
piisavalt neid ainulaadseid probleeme, ning on olemas lünk seotud riskide mõistmisel ja nende leevendamisel.
Käesoleva magistritöö eesmärk on kaardistada potentsiaalsed turvaprobleemid ja pakkuda leitud nõrkuste leevendamiseks strateegiaid ning
ületada lünk, viies läbi juhtumiuuringuid ja ekspert intervjuusid. See uurimus annab kaugosalus roboteid kasutavatele organisatsioonidele
parema arusaama turvariskidest ja tõhusatest lahendustest nende süsteemide ja kasutajate kaitsmiseks.
\vspace*{1ex}

\noindent\textbf{Keywords:}
\vspace{-1ex}
{\thesisKeywordsEst}
\vspace*{1ex}

\noindent\textbf{CERCS:}
\vspace{-1ex}
{\thesisCERCSEst}
