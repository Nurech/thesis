%! Author = partsjoo
%! Date = 02.04.2023

\newpage
\noindent\textbf{\large Cyber security risks in telepresence robotics and their mitigation}

\vspace*{3ex}

\noindent\textbf{Abstract:}
\vspace{-1ex}

Telepresence robotics (TPRs) have become increasingly popular, particularly
in higher education systems, as they enable users to remotely partake in
events. However, this increased usage also presents potential security risks
specific to TPRs, such as remote connection, cyber-physical presence, and
live video and audio feed. Current risk assessment models do not adequately
address these unique concerns, leading to a gap in understanding and
mitigating TPR-related risks. This thesis aims to develop a new risk
assessment model tailored to TPRs, incorporating existing frameworks such as
RSF, CIA, OCTAVE A, ISO27005, and NSMROS. By identifying and validating
potential risks through case studies and expert interviews, the study seeks
to propose mitigation strategies with an emphasis on user data security.
Ultimately, this research will provide organizations utilizing TPRs with a
better understanding of security risks and effective solutions to protect
their systems and users.

\vspace*{1ex}

\noindent\textbf{Keywords:} Cyber security, risk assessment, telepresence
robotics

\vspace*{1ex}

\noindent\textbf{CERCS:}

\vspace*{1ex}
