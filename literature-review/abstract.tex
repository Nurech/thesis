%! Author = partsjoo
%! Date = 02.04.2023

\newpage
\noindent\textbf{\large {\thesisTitle}}

\vspace*{3ex}

\noindent\textbf{Abstract:}
\vspace{-1ex}

\ac{tprs} have become increasingly popular, particularly
in higher education systems, as they enable users to remotely partake in
events. However, this increased usage also presents potential security risks
specific to \ac{tprs}, such as remote connection, cyber-physical presence, and
live video and audio feed. Current risk assessment models do not adequately
address these unique concerns, leading to a gap in understanding and
mitigating \ac{tprs} related risks.
This thesis aims to map potential security issues and how to mitigate them by incorporating existing
frameworks such as RSF, CIA, OCTAVE A, ISO27005, and NSMROS. By identifying and validating
potential risks through case studies and expert interviews, the study seeks
to propose mitigation strategies with an emphasis on user data security.
This research will provide organizations utilizing \ac{tprs} with a
better understanding of security risks and effective solutions to protect
their systems and users.

\vspace*{1ex}

\noindent\textbf{Keywords:}
\vspace{-1ex}

{\thesisKeywords}

\vspace*{1ex}


\noindent\textbf{CERCS:}
\vspace{-1ex}

{\thesisCERCS}

\vspace*{1ex}

\noindent\textbf{\large {\thesisTitleEst}}
\vspace*{1ex}

\noindent\textbf{Lühikokkuvõte:}
\vspace{-1ex}

Kaugkohalolu robotid on muutunud üha populaarsemaks, eriti kõrgemas haridussüsteemis, kuna need võimaldavad kasutajatel osaleda üritustel
kaugjuhtimise teel. Siiski kaasnevad selle suurenenud kasutamisega ka kaugkohalolu robotitele omased potentsiaalsed turvariskid, nagu
kaugühendus, küber-füüsiline kohalolek ning reaalajas video- ja heliside. Praegused riskihindamise mudelid ei käsitle
piisavalt neid ainulaadseid probleeme, ning on olemas lünk seotud riskide mõistmisel ja nende leevendamisel.
Käesoleva magistritöö eesmärk on kaardistada kaugkohalolu robotitega seotud turvariskid ning pakkuda ettepanekud seotud riskide
leevendamiseks, kasutades olemasolevaid raamistike ning riskihindamise mudeleid RSF, CIA, OCTAVE A, ISO27005 ja NSMROS riskide
tuvastamiseks. Täiendavalt toimub potentsiaalsete riskide tuvastamine ning valideerimine juhtumiuuringu (IT Kollež) näitel ja ekspertide
intervjuude abil püüab uuring pakkuda leevendusstrateegiaid, keskendudes kasutajate andmete turvalisusele. Lõppkokkuvõttes pakub see
teadustöö organisatsioonidele, kes kasutavad TPR-e, paremat arusaamist turvariskidest ja tõhusatest lahendustest oma süsteemide ja kasutajate kaitsmiseks.
